% !TeX spellcheck = en_GB
\documentclass[AERbeamer%              style
              ,optEnglish%            language
              %,handout%               deactivate animation
              ,optBiber%               bibliography tool
              ,optBibstyleAlphabetic%
              ,optBeamerClassicFormat% 4:3 format
              %,optBeamerWideFormat%   16:9 format
              ]{AERlatex}%
\setbeameroption{show notes}% Show all notes
%
% Set paths
\graphicspath{{figures_lecture_1/}}%
\addbibresource{literature.bib}%
%
% Package Imports
%\usepackage{media9}
\usepackage{graphicx}
\usepackage{multimedia}
\usepackage{subcaption}
\usepackage{amsmath}
%
% set meta data
\title{Probabilistic Programming for Scientific Discovery}%
\subtitle{Lecture 2}
\author{Ludger Paehler}% (optional)
\date{\AERutilsDate{29}{7}{2020}}% (optional)
\institute{Lviv Data Science Summer School}% (optional)
%
% Setup of header and footer
\AERbeamerSetupHeader{\AERlayoutHeaderCDChair}%
\AERbeamerSetupFooterCD%
%\AERbeamerSetupFooterSlideNumberOnly%
%
\begin{document}%
%
% Start with titlepage
\AERbeamerTitlePageDefault%
%
%\begin{frame}{Title of Slide}{Subtitle of Slide}%
%    \blindtext%
%\end{frame}%
%

% Slide 0: Table of Contents
\begin{frame}{Table of Contents}{}%
    % Contents
    \tableofcontents
\end{frame}%


% Approaches to Inference, the Core of every respectable Probabilistic Programming Framework
\section{Approaches to Inference - the Inference Engines}



% Probabilistic Programming Frameworks
\section{Probabilistic Programming Frameworks}


\subsection{Stan}
% Stan Slide 1: general introduction
\begin{frame}[c]{Stan \footnote{Carpenter, B., Gelman, A., Hoffman, M.D., Lee, D.,
                                Goodrich, B., Betancourt, M., Brubaker, M., Guo, J., Li, P.
                                and Riddell, A., 2017. Stan: A probabilistic programming
                                language. Journal of statistical software, 76(1).}}{Overview}
    \centering
    \begin{itemize}
        \item General overview of the purpose behind Stan
    \end{itemize}
\end{frame}


% Stan Slide 2: Example Code
\begin{frame}[c]{Stan}{Syntax}
    \centering
    \begin{itemize}
        \item Example code to get a grasp for the syntax
    \end{itemize}
\end{frame}


% Stan Slide 3: Application Performance
\begin{frame}[c]{Stan}{Application Performance}
    \centering
    \begin{itemize}
        \item Example applications
    \end{itemize}
\end{frame}


\subsection{Venture}
% Venture (2014) Slide 1: General Introduction
\begin{frame}[c]{Venture \footnote{Mansinghka, V., Selsam, D. and Perov, Y., 2014.
                                   Venture: a higher-order probabilistic programming platform
                                   with programmable inference. arXiv preprint arXiv:1404.0099.}}{Overview}
    \centering
    \begin{itemize}
        \item General overview of the purpose behind venture
    \end{itemize}
\end{frame}


% Venture Slide 2: Example Code
\begin{frame}[c]{Venture}{Syntax}
    \centering
    \begin{itemize}
        \item Example code to get a gauge for the syntax
    \end{itemize}
\end{frame}


% Venture Slide 3: Application Performance
\begin{frame}[c]{Venture}{Application Performance}
    \centering
    \begin{itemize}
        \item Application performance
    \end{itemize}
\end{frame}


\subsection{Anglican}
% Anglican (2014): General Introduction
\begin{frame}[c]{Anglican \footnote{Tolpin, D., van de Meent, J.W., Yang, H. and Wood, F., 2016, August. Design
                                    and implementation of probabilistic programming language anglican. In Proceedings
                                    of the 28th Symposium on the Implementation and Application of Functional
                                    programming Languages (pp. 1-12).}}{Overview}
    \centering
    \begin{itemize}
        \item General overview of the purpose behind Anglican
    \end{itemize}
\end{frame}


% Anglican Slide 2: Example Code
\begin{frame}[c]{Anglican}{Syntax}
    \centering
    \begin{itemize}
        \item Example code
    \end{itemize}
\end{frame}


% Anglican Slide 3: Application Performance
\begin{frame}[c]{Anglican}{Application Performance}
    \centering
    \begin{itemize}
        \item Application performance
    \end{itemize}
\end{frame}


\subsection{PyMC3}
% PyMC3 (2015) Slide 1: General Introdction
\begin{frame}[c]{PyMC3 \footnote{Salvatier, J., Wiecki, T.V. and Fonnesbeck, C., 2016. Probabilistic programming
                                 in Python using PyMC3. PeerJ Computer Science, 2, p.e55.}}{Overview}
    \centering
    \begin{itemize}
        \item General overview of the purpose behind PcMC3
    \end{itemize}
\end{frame}


% PyMC3 Slide 2: Example Code
\begin{frame}[c]{PyMC3}{Syntax}
    \centering
    \begin{itemize}
        \item Example code
    \end{itemize}
\end{frame}


% PyMC3 Slide 3: Application Performance
\begin{frame}[c]{PyMC3}{Application Performance}
    \centering
    \begin{itemize}
        \item Application performance
    \end{itemize}
\end{frame}


\subsection{TensorFlow Probability}
% TensorFlow Probability (2017) Slide 1: General Introduction
\begin{frame}[c]{TensorFlow Probability \footnote{Dillon, J.V., Langmore, I., Tran, D., Brevdo, E., Vasudevan, S.,
                                                  Moore, D., Patton, B., Alemi, A., Hoffman, M. and Saurous, R.A., 2017.
                                                  Tensorflow distributions. arXiv preprint arXiv:1711.10604.}}{Overview}
    \centering
    \begin{itemize}
        \item General overview of the purpose behind Tensorflow Probability
    \end{itemize}
\end{frame}


% Tensorflow Probability Slide 2: Example Code
\begin{frame}[c]{TensorFlow Probability}{Syntax}
    \centering
    \begin{itemize}
        \item Example code
    \end{itemize}
\end{frame}


% Tensorflow Probability Slide 3: Application Performance
\begin{frame}[c]{TensorFlow Probability}{Application Performance}
    \centering
    \begin{itemize}
        \item Application performance
    \end{itemize}
\end{frame}


\subsection{Pyro \& NumPyro}
% Pyro (2018) Slide 1: General introduction
\begin{frame}[c]{Pyro \footnote{Bingham, E., Chen, J.P., Jankowiak, M., Obermeyer, F., Pradhan, N., Karaletsos, T.,
                                Singh, R., Szerlip, P., Horsfall, P. and Goodman, N.D., 2019. Pyro: Deep universal
                                probabilistic programming. The Journal of Machine Learning Research, 20(1), pp.973-978.}
                \& NumPyro \footnote{Phan, D., Pradhan, N. and Jankowiak, M., 2019. Composable effects for flexible and
                                     accelerated probabilistic programming in NumPyro. arXiv preprint arXiv:1912.11554.}}{Overview}
    \centering
    \begin{itemize}
        \item General overview of the purpose behind Pyro \& NumPyro
    \end{itemize}
\end{frame}


% Pyro & NumPyro Slide 2: Example Code
\begin{frame}[c]{Pyro \& NumPyro}{Syntax}
    \centering
    \begin{itemize}
        \item Example code of Pyro \& NumPyro
    \end{itemize}
\end{frame}


% Pyro & NumPyro Slide 3: Application Performance  -> Might want to add a second slide here for the performance of NumPyro
\begin{frame}[c]{Pyro \& NumPyro}{Application Performance}
    \centering
    \begin{itemize}
        \item Application performance
    \end{itemize}
\end{frame}


\subsection{Edward2}
% Edward2 (2018/2019) Slide 1: General introduction
\begin{frame}[c]{Edward2 \footnote{Tran, D., Hoffman, M.W., Moore, D., Suter, C., Vasudevan, S. and Radul, A., 2018. Simple, distributed,
                                   and accelerated probabilistic programming. In Advances in Neural Information Processing Systems (pp. 7598-7609).}}{Overview}
    \centering
    \begin{itemize}
        \item General overview of the purpose behind Edward2
    \end{itemize}
\end{frame}


% Edward2 Slide 2: Example Code
\begin{frame}[c]{Edward2}{Syntax}
    \centering
    \begin{itemize}
        \item Example code of Edward2
    \end{itemize}
\end{frame}


% Edward2 Slide 3: Application Performance
\begin{frame}[c]{Edward2}{Application Performance}
    \centering
    \begin{itemize}
        \item Application performance
    \end{itemize}
\end{frame}


\subsection{Gen}
% Gen (2019) Slide 1: General introduction
\begin{frame}[c]{Gen \footnote{Cusumano-Towner, M.F., Saad, F.A., Lew, A.K. and Mansinghka, V.K., 2019, June. Gen: a general-purpose
                               probabilistic programming system with programmable inference. In Proceedings of the 40th ACM SIGPLAN
                               Conference on Programming Language Design and Implementation (pp. 221-236).}}{Overview}
    \centering
    \begin{itemize} % Mention the programmable inference to push the boundaries of prob prog here
        \item General overview of the purpose behind Gen
    \end{itemize}
\end{frame}


% Gen Slide 2: Example Code
\begin{frame}[c]{Gen}{Syntax}
    \centering
    \begin{itemize}
        \item Example code of Gen
    \end{itemize}
\end{frame}


% Gen Slide 3: Application perforance
\begin{frame}[c]{Gen}{Application Performance}
    \centering
    \begin{itemize}
        \item Application performance
    \end{itemize}
\end{frame}


\subsection{PyProb}
% PyProb (2019) Slide 1: General introduction
\begin{frame}[c]{PyProb \footnote{Baydin, A.G., Shao, L., Bhimji, W., Heinrich, L., Naderiparizi, S., Munk, A., Liu, J., Gram-Hansen, B.,
                                  Louppe, G., Meadows, L. and Torr, P., 2019. Efficient probabilistic inference in the quest for physics
                                  beyond the standard model. In Advances in neural information processing systems (pp. 5459-5472).}}{Overview}
    \centering
    \begin{itemize}
        \item General overview of the purpose behind PyProb
    \end{itemize}
\end{frame}


% PyProb Slide 2: Example Code
\begin{frame}[c]{PyProb}{Syntax}
    \centering
    \begin{itemize}
        \item Example code of PyProb
    \end{itemize}
\end{frame}


% PyProb Slide 3: Application performance
\begin{frame}[c]{PyProb}{Application Performance}
    \centering
    \begin{itemize}
        \item Application performance
    \end{itemize}
\end{frame}


\subsection{Turing}
% Turing (2018) Side 1: General introduction
\begin{frame}[c]{Turing \footnote{Ge, H., Xu, K. and Ghahramani, Z., 2018, March. Turing: A Language for Flexible
                                  Probabilistic Inference. In International Conference on Artificial Intelligence
                                  and Statistics (pp. 1682-1690).}}{Overview}
    \centering
    \begin{itemize}
        \item General overview of the purpose behind Turing
    \end{itemize}
\end{frame}


% Turing Slide 2: Example code
\begin{frame}[c]{Turing}{Syntax}
    \centering
    \begin{itemize}
        \item Example code of Turing
    \end{itemize}
\end{frame}


% Turing Slide 3: Application performance
\begin{frame}[c]{Application performance}
    \centering
    \begin{itemize}
        \item Application performance
    \end{itemize}
\end{frame}


% Summary in matrix form
\begin{frame}[c]{Probabilistic Programming Frameworks}{Summary}
    \centering
    \begin{itemize}
        \item Summary of all probabilistic programming frameworks in a single table
    \end{itemize}
\end{frame}



% Practical Introduction to Turing
\section{Practical Introduction to a Probabilistic Programming Framework}



% Extension to a more complex example
\section{Extending the ideas to a more complex examples}







%
%\AERbeamerSetFooterText{References}%
%\begin{frame}[allowframebreaks]{References}%
%    \printbibliography[heading=none]%
%\end{frame}%
%
% End with titlepage
%\AERbeamerTitlePageDefault%
%
%
\end{document}%
%
%
